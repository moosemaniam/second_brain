\documentclass[recipes.tex]{subfiles}
\begin{document}
\section{ಕಾಯಿ ಸಾಸಿವೆ ಚಿತ್ರಾನ್ನ}
{\small ನಾಲಕ್ಕು ಜನರಿಗೆ ಆಗೋಹಾಗೆ }
\subsection{ಸಾಮಗ್ರಿಗಳು}
\subsubsection{ಕಾಯಿ ಸಾಸಿವೆ ಗೊಜ್ಜು}
{
    \small 
\begin{enumerate}

    \item ಮುಕ್ಕಾಲು ಟೀ ಸ್ಪೂನ್ ಸಾಸಿವೆ ಕಾಳು 
    \item ಮುಕ್ಕಾಲು ಕಪ್ ಕೊಬ್ಬರಿ ತುರಿ 
    \item 4 ಇಂದ 6 ಬ್ಯಾಡಗಿ ಅಥವಾ ಗುಂಟೂರು ಮೆಣಸಿನಕಾಯಿ  
    \item ಹುಣಸೇ ಹಣ್ಣು : ನಲ್ಲಿ ಕಾಯಿ ಗಾತ್ರದ್ದು  
    \item ಬೆಲ್ಲ : ನಲ್ಲಿ ಕಾಯಿ ಗಾತ್ರದ್ದು  
    \item ಉಪ್ಪು : ರುಚಿಗೆ ತಕ್ಕಷ್ಟು
\end{enumerate}
\subsubsection{ಒಗ್ಗರಣೆ}
\begin{enumerate}
    \item ಕೊಬ್ಬರಿ ಎಣ್ಣೆ 2 ಟೀ ಚಮಚ
    \item ಸಾಸಿವೆ 
    \item ಉದ್ದಿನ ಬೇಳೆ
    \item ಕಡಳೆ  ಬೇಳೆ 
    \item ಇಂಗು
    \item ಒಂದು ಒಣ ಮೆಣಸಿನ ಕಾಯಿ ಮುರುದು
    \item ಅರುಶಿನ
    \item ಕಡಲೇ  ಕಾಯಿ 
    \item ಕರಿಬೇವು ಸೊಪ್ಪು 
\end{enumerate}
\subsubsection{ಮಾಡುವ ಪರಿ}
ಅಕ್ಕಿಯನ್ನು ಚೆನ್ನಾಗಿ ತೊಳೆದು ಉದುರುದುರು ಆಗೊಹಾಗೆ ಬೆಯಿಸಿಕೊಳ್ಳಿ. ಓಂದು ತಟೆಯಲ್ಲಿ ಹಾಕಿ ಆರಿಸಿಕೊಳ್ಳಿ.
ಒಂದು ಕಬ್ಬಿಣದ ಬಾಂಡಲೆಯಲ್ಲಿ ಒಗ್ಗರಣೆ ಹಾಕಿ. ಗೊಜ್ಜನ್ನು ಇದಕ್ಕೆ ಹಾಕಿ ಎಣ್ಣೆ ಬಿಡುವವಾರೆಗೂ ಕರೆಯಿರಿ.
ಗೊಜ್ಜನ್ನು ಅನ್ನಕ್ಕೆ ಸ್ವಲ್ಪ ಸ್ವಲ್ಪ ಹಾಕಿ ಅನ್ನ ಮುದ್ದೆ ಆಗದ ಹಾಗೆ ಕಲಸಿಕೊಳ್ಲಿ. ಕಲಾಸುತ್ತಾ ರುಚಿಗೆ ತಕ್ಕಂತೆ ಉಪ್ಪು ಹಾಗು 
ಗೊಜ್ಜು ಹಾಕುತ್ತಾ ಹೊಗಿ.
}
\end{document}



